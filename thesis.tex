%%%%%%%%%%%%%%%%%%%%%%%%%%%%%%%%%%%%%%%%%%%%%%%%%%%
%%  复旦大学(硕士)学位论文模版 (UTF-8/XeLaTex)
%%  2013-03 richard@bbs.fudan.edu.cn
%%%%%%%%%%%%%%%%%%%%%%%%%%%%%%%%%%%%%%%%%%%%%%%%%%%
\documentclass{fudan}

\title{论文标题}
\author{
指导小组成员名单\vspace{1em}\\
某\hspace{1em}人 \hspace{1em} 教\hspace{1em}授\\
某某人 \hspace{1em} 副教授\\
}
\date{}

\begin{document}

%% 也可以选择使用下面的命令产生封面,需要fudan.png文件
\makefrontcover{10246}{1020124XXXX}%
{论文标题}%
{计算机科学技术学院}%
{计算机软件与理论}%
{某学生}%
{某某人\hspace{1.5em}教授}%
{2013年4月22日}

%% 扉页
\maketitle

%% 目录
\tableofcontents

%% 去掉摘要章节的章节编号
\frontmatter

%% 中文摘要
\chapter{摘\hspace{1.5em}要}

\textbf{【关键词】}

\textbf{【中图分类号】} TP183

%% 英文摘要
\chapter{Abstract}

\textbf{Keywords:}

\textbf{CLC Number:} TP183

%% 正文从这里开始
\mainmatter

\chapter{章的标题}

\section{节的标题}

\subsection{小节的标题}

内容。

段首空格。

%% 一个表格
\begin{table}[!h]
\begin{center}
\begin{tabular}{r|c}
\hline\hline
方法 & 结果 \\
\hline
他们的 & 还可以 \\
你们的 & 还可以 \\
我们的 & 非常非常的好啊 \\
\hline\hline
\end{tabular}
\end{center}
\caption{实验结果}
\end{table}

%% 一张图片
\begin{figure}[!h]
\begin{center}
\fbox{\rule{0pt}{2in} \rule{0.9\linewidth}{0pt}}
\end{center}
\caption{一个图示}
\end{figure}

\nocite{*} %% 列出全部(包括未引用的)参考文献

\begin{appendix}

\chapter{附录}

\end{appendix}

%% 去掉附录部分的章节编号
\backmatter

%% 将参考文献列入目录 (MARK 注释掉了,不在需要)
%% MARK \addcontentsline{toc}{chapter}{参考文献}
\bibliography{ref}

\chapter{致\hspace{1.5em}谢}

\makebackcover

\end{document}
